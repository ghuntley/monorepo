\documentclass[12pt]{beamer}
\usetheme{metropolis}
\newenvironment{code}{\ttfamily}{\par}
\title{Where does \textit{your} compiler come from?}
\date{2018-03-13}
\author{Vincent Ambo}
\institute{Norwegian Unix User Group}
\begin{document}
  \maketitle

  %% Slide 1:
  \section{Introduction}


  %% Slide 2:
  \begin{frame}{Chicken and egg}
    Self-hosted compilers are often built using themselves, for example:

    \begin{itemize}
    \item C-family compilers bootstrap themselves \& each other
    \item (Some!) Common Lisp compilers can bootstrap each other
    \item \texttt{rustc} bootstraps itself with a previous version
    \item ... same for many other languages!
    \end{itemize}
  \end{frame}

  %% Slide 3:
  \begin{frame}{Trusting Trust}
    \begin{center}
      Could this be exploited?
    \end{center}
  \end{frame}

  %% Slide 4:
  \begin{frame}{Short interlude: A quine}
    \begin{center}
      \begin{code}
        ((lambda (x) (list x (list 'quote x)))
        \newline\vspace*{6mm} '(lambda (x) (list x (list 'quote x))))
      \end{code}
    \end{center}
  \end{frame}

  %% Slide 5:
  \begin{frame}{Short interlude: Quine Relay}
    \begin{center}
      \includegraphics[
        keepaspectratio=true,
        height=\textheight
      ]{quine-relay.png}
    \end{center}
  \end{frame}

  %% Slide 6:
  \begin{frame}{Trusting Trust}
    An attack described by Ken Thompson in 1984:

    \begin{enumerate}
    \item Modify a compiler to detect when it's compiling itself.
    \item Let the modification insert \textit{itself} into the new compiler.
    \item Add arbitrary attack code to the modification.
    \end{enumerate}
  \end{frame}

  %% Slide 7:
  \begin{frame}{Damage potential?}
    \begin{center}
      Let your imagination run wild!
    \end{center}
  \end{frame}
\end{document}
