\documentclass[14pt]{beamer}
\usetheme{metropolis}

\title{systemd}
\subtitle{The standard Linux init system}

\begin{document}
\metroset{titleformat frame=allcaps}

\maketitle

\section{Introduction}

\begin{frame}{What is an init system?}
  An init system is the first process (PID 1) started in a Unix like system. It handles:

  \begin{itemize}
  \item Starting system processes and services to prepare environment
  \item Adopting and ``reaping'' orphaned processes
  \end{itemize}
\end{frame}

\begin{frame}{Classical init systems}
  Init systems before systemd - such as SysVinit - were very simple.

  \begin{itemize}
  \item Services and processes to run are organised into ``init scripts''
  \item Scripts are linked to specific runlevels
  \item Init system is configured to boot into a runlevel
  \end{itemize}

\end{frame}

\section{systemd}

\begin{frame}{Can we do better?}
  \begin{itemize}
  \item ``legacy'' init systems have a lot of drawbacks
  \item Apple is taking a different approach on OS X
  \item Systemd project was founded to address these issues
  \end{itemize}
\end{frame}

\begin{frame}{Systemd design goals}
  \begin{itemize}
  \item Expressing service dependencies
  \item Monitoring service status
  \item Enable parallel service startups
  \item Ease of use
  \end{itemize}
\end{frame}

\section{Demo}

\section{Controversies}

\section{Questions?}

\end{document}
